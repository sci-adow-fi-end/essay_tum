\documentclass{article}
\usepackage{graphicx}
\usepackage{background}
\usepackage{tikz}
\usepackage{filecontents}
\renewcommand{\baselinestretch}{1.5}
\usepackage[margin=2cm]{geometry}
\usepackage{times}



\title{\LARGE The role of Artificial Intelligence in future technology\\
  
  \large and the effect of increasing processing power on it}
\author{Alessio Delli Colli}
\date{May 2024}

\backgroundsetup {
   scale=1,
   angle=0,
   opacity=1,
   color=black,
   contents={\begin{tikzpicture}[remember picture, overlay]
       \node at ([xshift=+2cm,yshift=-1cm] current page.north west)
            {\includegraphics[width = 3cm]{tum.png}};
     \end{tikzpicture}}
 }

\begin{document}

\maketitle

\newpage



\section{Introduction and literature analysis}
The future of AI is one of the most debated topics due to the high
relevance of assessing its potential and consequences.
In this communication so will be attempted by observing AI's history and
analyzing how it evolved to the modern days.
The limited number of Contributions found on this topic and their various apparent quality, as well
as the novelty and the tumultuous growth of the applications, suggest that the topic is far to be fully
addressed and further investigations are worthwhile.
Notwithstanding, the available literature \cite{futofai1}\cite{futofai2}
\cite{futofai3} allows to point out several major observations and remarks.

\begin{itemize}
\item
  Several subfields of AI exist; they are strongly application dependent. On the other hand, some applications would not exist at all without AI.

\item
  AI is believed to have the potential to change major aspects of economy, society and life, although some extreme claim can be found that is yet
  disproportional and/or unfounded. In fact, while “visionary research” is auspicated in connection with application specific projects to advance
  the AI, warnings have been issued to apart 'science fiction' from practical reality.

\item
  AI is credited to present significant advantages, although it made ethical and legal possible problems arise, which are still to be addressed.

\item
  Past AI development incurred in successes and failures, which are considered significant in order to comprehend further steps.

\item
  AI market is expected to size close to $\$$170 Billions in 2025.

\item
  Different AI formal definitions exist.

\end{itemize}

\newpage

\section{Brief history of Ai}

As stated in \cite{AIhistory}, AI has been a long debated topic trough recent history.
Its first applications were found in the field of translation
between 1950 and 1960.
The early interest in the topic culminated with the invention
of "The Perceptron" in 1957 by Frank Rosenblatt.
It was a machine
inspired by the human brain that we can consider the precursor of
modern neural networks.
This enthusiasm ended in 1969 with the article "Perceptrons. An
Introduction to Computational Geometry" by Minsky and Pappert that showed
the poor performance of perceptrons in many common tasks.
This date has been considered the start of the first "AI winter".
The interest in this topic arose again in the early 1980s
whit the commercialization of expert systems began.
This commercialization has been accompanied by many promises about
these systems capabilities that were clearly overestimated.
In fact, many experts at the time criticized the capabilities of
these solutions; for example John McCarthy in 1884
stated that expert systems lacked common sense
and knowledge about their own limitations, and showed they were
not ready for critical applications like the medical field.
These problems led to the second "ai winter" that lasted
from the late 1980s to the early 2000s.
In the modern days the focus has moved from expert systems
to a machine learning oriented approach and, thanks to the
intuition of constructing multi-layered neural networks, we have
seen the rise of techniques like deep learning and generative models.


\section{Current applications}

Thanks to the analysis taken out in \cite{AIsubfields} we can observe
that  many of the techniques mentioned above have found,
and still occupy, a significant place in modern AI Research and
applications.
Machine learning seems to be the topic that raises more interest between
researchers, which often focus on techniques based on artificial
neural networks like deep learning.
Deep learning techniques are widely used for example in the medical field
where they provide mostly diagnostic and treatment decision support.
Other areas that are gaining an increasing popularity in last years are
Computer vision and Natural language processing.
Both techniques concur in obtaining a more natural interaction
between man and machine and will be fundamental in applications like
automated transport or AI assistance of people with disabilities.

\section{AI and  computational complexity}

The development of the above-mentioned techniques has been possible thanks
to the increase in hardware performance given by the effect of the Moore's
law and the introduction of specific AI acceleration devices.
Currently, a further similar progression has to be considered rather
uncertain.. 
in the near future, in order to get more and more precise, machine learning
tools will in fact require more and more processing power.
The predictions presented in \cite{comp}  show that some of these models to get the
best results will require between $10^{50}$ and $ 10^{120} $ flops of
processing power.
These data make clear that what is needed is not only hardware that can
withstand this kind of load but also hardware that can do it efficiently
to avoid the economic and environmental concerns that arise from such power
requirements.

\section{Conclusion}

The data available nowadays clearly states that artificial intelligence and
many of its sub fields will have a fundamental role in the development of
the near future's technology thanks to the flexibility they enable in the
invention of new solutions but it's important to keep in mind that the
advancements in AI techniques are, and probably will always be, bound to the
hardware they run on and disproportional advancements in algorithms and
hardware have many times led to periods of stagnation.



\bibliographystyle{IEEEtran}
\bibliography{BIB}

\newpage
Declaration of Authorship\\
I hereby declare that the essay submitted is my own unaided work.
All direct or indirect sources used are
acknowledged as references.\\
I am aware that the essay in digital form can be examined for the use
of unauthorized aid and in order to
determine whether the essay as a whole or parts incorporated in it may
be deemed as plagiarism. For the
comparison of my work with existing sources I agree that it shall be
entered in a database where it shall
also remain after examination, to enable comparison with future theses
submitted. Further rights of
reproduction and usage, however, are not granted here.\\
This paper was not previously presented to another examination board and has not been published.


\end{document}
